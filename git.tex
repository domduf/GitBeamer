\documentclass[10pt]{beamer}
\usetheme{Montpellier}
\usepackage[utf8]{inputenc}
\usepackage[french]{babel}
\usepackage[T1]{fontenc}
\usepackage{amsmath}
\usepackage{hyperref}
\usepackage{amsfonts}
\usepackage{amssymb}
\usepackage{graphicx}
\graphicspath{{images/}}
\author{Dominique DUFOUR domduf.com}
\title{Git}
%\setbeamercovered{transparent} 
%\setbeamertemplate{navigation symbols}{} 
%\logo{} 
%\institute{} 
%\date{} 
%\subject{} 
\begin{document}

\begin{frame}
\titlepage
\end{frame}

\begin{frame}
\tableofcontents
\end{frame}

\begin{frame}{Qu'est-ce que c'est, à quoi ça sert, qui qu'a fait ça ?}
\framesubtitle{parce que ça n'a pas l'air simple...}

Lorsque le noyau de Linux devint un peu compliqué à faire évoluer (un grand nombre de développeurs étaient sur le coup, maintenir un historique des versions commençait à devenir chaud), un homme très important, \textbf{Linus Torvalds}, le créateur de Linux se retroussa les manches et créa ce formidable outil.

---

Voilà déjà un petit lien si vous êtes perdus ou curieux.
\\
lien en dur: \url{https://git-scm.com/book/fr/v2/}

\href{https://git-scm.com/book/fr/v2/}{la version 2 du book}

\end{frame}



\begin{frame}{L'intérêt de Git.}{sinon c'est pas la peine...}

Si vous avez déjà travaillé sur un projet,, tel la construction d'un site internet, \textbf{avec une autre personne}, vous vous êtes 
certainement déjà retrouvé avec ce genre de chose:

\includegraphics[width=150pt]{SiteAvecUneAUtrePersonne.png}
C'est à dire un répertoire rempli de dossiers du même site, ayant un nom de version différent, et au bout du compte...\textbf{bordélique}.

\end{frame}



\begin{frame}
Git va permettre de faire plusieurs choses \textbf{très intéressantes}:
\begin{itemize}
\item Permettre d'utiliser \textbf{un seul dossier}
\end{itemize}
\includegraphics[width=200pt]{SiteAvecUneAUtrePersonneGit.png}

\begin{itemize}
\item sauvegarder les anciennes versions du site, c'est à dire tout le boulot précédant, dans ce même dossier, \textbf{de façon transparente}.
\item \textbf{partager} votre version avec votre binôme, en ligne.
\item travailler sur une partie du site, pendant que votre binôme s'occupe d'une autre, et ensuite "mélanger" votre travail, de façon \textbf{non destructive}.
\item 
\end{itemize}

\end{frame}



\begin{frame}
La suite au prochain épisode...
\end{frame}

\end{document}
