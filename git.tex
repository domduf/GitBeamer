\documentclass[10pt]{beamer}
\usetheme{Montpellier}
\usepackage[utf8]{inputenc}
\usepackage[french]{babel}
\usepackage[T1]{fontenc}
\usepackage{amsmath}
\usepackage{hyperref}
\usepackage{amsfonts}
\usepackage{amssymb}
\usepackage{graphicx}
\graphicspath{{images/}}
\author{Dominique DUFOUR domduf.com}
\title{Git}
%\setbeamercovered{transparent} 
%\setbeamertemplate{navigation symbols}{} 
%\logo{} 
%\institute{} 
%\date{} 
%\subject{} 




\begin{document}

\begin{frame}
\titlepage
\end{frame}

\begin{frame}{table}
\tableofcontents
\end{frame}

\section{Présentation}

\begin{frame}{Qu'est-ce que c'est, à quoi ça sert, qui qu'a fait ça ?}

\framesubtitle{parce que ça n'a pas l'air simple...}

Lorsque le noyau de Linux\cite{noauthor_git_nodate} devint un peu compliqué à faire évoluer (un grand nombre de développeurs étaient sur le coup, maintenir un historique des versions commençait à devenir chaud), un homme très important, \textbf{Linus Torvalds}, le créateur de Linux se retroussa les manches et créa ce formidable outil.

---

Voilà déjà un petit lien si vous êtes perdus ou curieux.
\\

\href{https://git-scm.com/book/fr/v2/}{la version 2 du git-book}: \url{https://git-scm.com/book/fr/v2/}



\end{frame}



\begin{frame}{L'intérêt de Git.}{sinon c'est pas la peine...}

Si vous avez déjà travaillé sur un projet, tel la construction d'un site internet, \textbf{avec une autre personne}, vous vous êtes 
certainement déjà retrouvé avec ce genre de chose:

\includegraphics[width=150pt]{SiteAvecUneAUtrePersonne.png}
C'est à dire un répertoire rempli de dossiers du même site, ayant un nom de version différent, et au bout du compte...\textbf{bordélique}.

\end{frame}



\begin{frame}
Git va permettre de faire plusieurs choses \textbf{très intéressantes}:
\begin{itemize}
\item Permettre d'utiliser \textbf{un seul dossier}
\end{itemize}
\includegraphics[width=200pt]{SiteAvecUneAUtrePersonneGit.png}

\begin{itemize}
\item sauvegarder les anciennes versions du site, c'est à dire tout le boulot précédant, dans ce même dossier, \textbf{de façon transparente}.
\item \textbf{partager} votre version avec votre binôme, en ligne.
\item travailler sur une partie du site, pendant que votre binôme s'occupe d'une autre, et ensuite "mélanger" votre travail, de façon \textbf{non destructive}.
\item 
\end{itemize}

\end{frame}

\begin{frame}

\end{frame}


\section{Installer}

\begin{frame}
Pour tous les systèmes, lisez d'abord la page ici:
\url{https://git-scm.com/book/fr/v2} à \textbf{la section 1.5}
\end{frame}

\begin{frame}{Linux}

\end{frame}


\section{Utiliser en local}


\begin{frame}{Utiliser en local}
Lorsqu'on parle d'utilisation en local, cela concerne votre travail sur votre propre ordinateur.

Sans pour l'instant parler de partage de votre projet avec votre ou vos collaborateurs.

Vous disposez d'un dossier, par exemple un dossier de site web nommé \textbf{monSite}, qui se trouve dans le dossier \textbf{htdocs} de votre serveur local xampp ( ou lampp sous Linux):

\includegraphics[width=180pt]{monSite.png}
Vous allez ouvrir une console (sous Windows, un clic droit sur le dossier \textbf{monSite} va vous proposer l'option d'ouverture de la console, si git est installé)
\end{frame}

\begin{frame}


\end{frame}

\subsection{les commandes indispensables}

\begin{frame}{les commandes indispensables}
\begin{itemize}
\item git status

Permet de \textbf{visualiser} les fichiers qui ont été modifiés, ajoutés, supprimés, depuis le dernier commit

\item git add

Permet d'\textbf{ajouter} les fichiers qui devront être \textbf{suivis} dans leur version lors du prochain commit.

\item git commit

Permet de faire \textbf{rentrer dans l'historique} tous les fichiers "marqués" par le git add
\item git log

Va afficher un historique des commits déja réalisés.
\includegraphics[width=150pt]{git-log.png}


\end{itemize}
\end{frame}


\subsection{les alias}

\begin{frame}{Les alias}

Les alias sont des raccourcis de commande\footnote{\url{https://github.com/domduf/git_trucs_et_astuces/blob/master/alias_utiles.txt}}.

Ces raccourcis sont appliqués bien sûr aux commandes les plus utilisées, et vont vous faire gagner du temps.

On les définit soi-même (une foi pour toute) en ligne de commande, en tapant:

 \textbf{git config --global alias.nomDeLAlias commande} 
 
 \smallbreak
 
Par exemple, pour utiliser \textbf{git st} au lieu de \textbf{git status} , la
commande de création de l'alias sera:

\textbf{git config --global alias.st status}\\

\includegraphics[width=150pt]{git-status.png} \includegraphics[width=150pt]{git-st.png}



\end{frame}

\subsection{le fichier .gitignore}
\subsection{Premiers Commits}


\section{Le dépôt sur Github}

\subsection{Mise à jour / fetch}
\subsection{Pousser / push}
\subsection{Tirer / pull}


\section{Les branches}

\subsection{Politique de travail }
\begin{frame}{Politique de travail  \footnote{https://bioinfo-fr.net/git-usage-collaboratif}}


\begin{itemize}
  \item vous faites une branche pour ajouter une nouvelle fonctionnalité
  
  \item  en plein milieu on vous demande de corriger un bug en urgence
  
   \item     vous ne le faites pas dans votre nouvelle branche car le code n'est pas encore fonctionnel
   
        vous ne le faites pas non plus dans la branche principale car celle-ci ne doit pas être utilisée comme branche de développement
        
  \item          vous créez une seconde branche depuis la branche principale pour corriger le bug
  
   \item         vous fusionnez cette seconde branche avec la branche principale une fois la correction terminée
  
  \item          vous terminez votre ajout de fonctionnalité sur la première branche
    
    \item        vous fusionnez votre première branche avec la (nouvelle) branche principale en réglant les conflits éventuels
\end{itemize}
 

\end{frame}

\section{Mettre en commun/Merger}


\section{The End}


\begin{frame}
La suite au prochain épisode... 
pour les curieux, le code de ce document est là:

\href{https://github.com/domduf/GitBeamer}{https://github.com/domduf/GitBeamer}
\end{frame}





\end{document}
